\documentclass{wluckin_article}

\begin{document}

\wtitle{Test Document: Parks in the USA}{Will Luckin}{\today}
\maketitle
\thispagestyle{empty}

\section{About Parks}
A park is an area of \textit{natural, semi-natural, or planted space} set
aside for human enjoyment and recreation or for the protection of wildlife or
natural habitats. It may consist of grassy areas, rocks, soil, and trees, but
may also contain buildings and other artifacts such as monuments, fountains or
playground structures.

In North America, many parks have fields for playing
sports such as soccer, baseball and football, and paved areas for games such
as basketball. Many parks have trails for walking, biking and other
activities.

Some parks are built adjacent to bodies of water or watercourses
and may comprise a beach or boat dock area. Often, the smallest parks are in
urban areas, where a park may take up only a city block or less. Urban parks
often have benches for sitting and may contain picnic tables and barbecue
grills. Parks have differing rules regarding whether dogs can be brought into
the park: some parks prohibit dogs; some parks allow them with restrictions
(e.g., use of a leash); \textit{and some parks, which may be called ``dog
  parks'', permit dogs to run off-leash.}

\section{Additional Information about Parks}
\subsection{What happens in the case of templating?}
The largest parks can be vast natural areas of hundreds of thousands of square
kilometres (thousands of square miles), with abundant wildlife and natural
features such as mountains and rivers. In many large parks, camping in tents
is allowed with a permit.

\subsection{What happens if you run out of ideas for subsections?}
\subsubsection{I'm so sorry}
Many natural parks are protected by law, and users may have to follow
restrictions \textit{(e.g., rules against open fires or bringing in glass
bottles}). Large national and sub-national parks are typically overseen by a
park ranger or a park warden. Large parks may have areas for canoeing and
hiking in the warmer months and, in some northern hemisphere countries,
cross-country skiing and snowshoeing in colder months.

\begin{figure}[H]
  \centering
  \includegraphics[width=\textwidth]{park}
  \caption{An example park}
  \label{fig:park}
\end{figure}

As can be seen in \textlcsc{Figure \ref{fig:park}}, the largest parks can be
vast natural areas of hundreds of thousands of square kilometres (thousands of
square miles), with abundant wildlife and natural features such as mountains
and rivers. In many large parks, camping in tents is allowed with a permit.

\subsubsection{But what about subsubsections?!}
Yes, they're included too.

\end{document}

\section{Praise Satan}
Hello world!

\end{document}

%%% Local Variables: 
%%% coding: utf-8
%%% mode: latex
%%% TeX-engine: luatex
%%% End: 
